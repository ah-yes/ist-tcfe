\section{Introduction}
\label{sec:introduction}

% state the learning objective

In this assignment we present and analyse a possible design for an audio amplifier circuit, with an input audio of maximum 10mV and supply voltage VCC at 12V. Thus, we used a Gain Stage composed by commom emitter amplifier and an output stage composed by common collector amplifier.
Furthermore, a cost/effectiveness analysis was also done, with set prices for the transistors (0.1 monetary unit per transistor), resistors (1 monetary unit per kOhm) and capacitors(1 monetary unit per micro-Farad).

We used the following components 

\begin{figure}[h] \centering
\includegraphics[scale=0.5]{lab04.pdf}
\caption{Audio amplifier circuit.}
\label{fig:rc}
\end{figure}

\begin{table}[h]
  \centering
  \begin{tabular}{|l|r|}
    \hline    
    {\bf Name} & {\bf Value [A or V]} \\ \hline
    $C_{coupling}$ & 1$\mu F$ \\ \hline
    $R_{B1}$ & 80k$\Omega$ \\ \hline
    $R_{B2}$ & 10k$\Omega$ \\ \hline
    $R_{C2}$ & 0.9k$\Omega$ \\ \hline
    $R_{E1}$ & 70$\Omega$ \\ \hline
    $C_{E1}$ & 80$\mu F$ \\ \hline
    $R_{out}$ & 80$\Omega$ \\ \hline
    $C_{out}$ & 120$\mu F$ \\ \hline  
  \end{tabular}
  \caption{Values for the components used}
  \label{tab:op}
\end{table}

leading to a cost of 165.17 MU. 


We then began by analysing the circuit by means of simulation with ngspice and a theoreti-
cal approach using Octave. Afterwards the results from both methods were compared followed
by a discussion of the efficiency of the speaker.
