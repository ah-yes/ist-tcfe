\section{Theoretical Analysis}
\label{sec:analysis}

For the theoretical analysis, we considerer an ideal operational amplifier, i.e., wiht infinite gain, infinite input impedence and zero output impedence. We also assume the OP-AMP is not saturated. In this conditions, the voltage at the minus and plus terminal are the same and no current flows into either.

At the plus terminal we have a simple voltage divider, thus 
\begin{equation}
\frac{\tilde{V_{+}}}{\tilde{V_{in}}} = \frac{j\omega C_1R_1}{R_1j\omega C_1 + 1}
\end{equation}

By applying Kirchoff's laws we arrive at an equation for the voltage at the output, 
\begin{equation}
\frac{\tilde{V_{out}}}{\tilde{V_{+}}} = \bigg(1+\frac{R_4}{R_3}\bigg)\frac{1}{R_2j\omega C_2 + 1}
\end{equation}

Hence, combining the two with  get 
\begin{equation}
\text{Gain} = \frac{\tilde{V_{out}}}{\tilde{V_{in}}} =\bigg(1+\frac{R_4}{R_3}\bigg)\frac{1}{R_2j\omega C_2 + 1}\frac{j\omega C_1R_1}{R_1j\omega C_1 + 1}
\label{eq:Gain}
\end{equation}

\begin{figure}[!htb]
\centering
\includegraphics[width=0.7\textwidth]{../mat/Gain.png}
\caption{Frequency response of the circuit}
\label{fig:venvelope}
\end{figure}

In particular for $f = 1.004690e+03\text{Hz}$\footnote{This frequency corresponds to the central obtained by the simulation} we have 
%%%%Insert output for op from octave
$$ \text{Gain} = 36.563331$$ 




